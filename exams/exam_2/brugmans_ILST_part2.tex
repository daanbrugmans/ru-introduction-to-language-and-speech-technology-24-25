\documentclass{Interspeech2024}

% 2024-11-25 modified by Cristian Tejedor-García (cristian.tejedorgarcia@ru.nl) 

% 2023-10-21 modified by Simon King (Simon.King@ed.ac.uk)  

% 2024-01 modified by TPC Chairs of Interspeech 2024  

% Keep this uncommented
\interspeechcameraready 


% **************************************
% *                                    *
% *      STOP !   DO NOT DELETE !      *
% *          READ THIS FIRST           *
% *                                    *
% * This template also includes        *
% * important INSTRUCTIONS that you    *
% * must follow when preparing your    *
% * paper. Read it BEFORE replacing    *
% * the content with your own work.    *
% **************************************

% Clearly state the focus of your chosen topic (add some extra information based on the topic/subtopic/categories that you want to focus).
\title{Max 4-Page (+1 for Refs.) Project's Template for the ILST Course}


% Full name, Radboud number and email. 
\name[affiliation={1}]{Daan}{Brugmans}
%\name[affiliation={2}]{FirstNameB}{LastNameB}
%\name[affiliation={3}]{FirstNameC}{LastNameC}


% if you have too many addresses to fit within the available space, try removing the "\\" newlines
\address{
  $^1$S1080742}
  %^2$RU numberB \\
  %$^3$RU numberC}
\email{daan.brugmans@ru.nl}%, B@ru.nl, c@ru.nl}

% min. 3 and max. 5
\keywords{keyword1, keyword2, keyword3, keyword4, keyword5}

\newcommand{\red}[1]{\textcolor{red}{#1}}

\begin{document}

\maketitle


\begin{abstract}   
    1000 characters maximum. ASCII characters only. No citations. Provide a concise summary of your project, including key objectives and findings. 
\end{abstract}



% Max. 4 pages (+1 extra for References only)
\section{Introduction}

\begin{itemize}
    \item Clearly outline the relevance and significance of your chosen topic.
    \item State the \textbf{research question (RQ)} guiding your manuscript.
    \item Tip: References to general reports and literature reviews are welcomed in this section.
\end{itemize}


\section{Method}

\subsection{Search Procedure}

\begin{itemize}
    \item Explain in detail how you carried out the search of related papers: Databases used, keywords, number of papers found/discarded/filtered, exclusion and inclusion criteria and any limitation related to your search procedure.
    \item Minimum 5 and maximum 10 scientific papers need to be analyzed in your project.
    \item Remember that the final selected papers must have been published 2019 onwards.
\end{itemize}






\subsection{Models}
Briefly describe the relevant information related to speech-based models of the papers found in a well-organized way.


\subsection{Datasets}
Briefly describe the relevant information related to the speech-based datasets used in the papers found in a well-organized way.

\subsection{Metrics}
Briefly describe the relevant information related to the results metrics used in the papers found in a well-organized way.


\section{Results}
\begin{itemize}
    \item Describe objectively (without personal thoughts) the results reported in at least five related papers (and maximum ten papers). A clear organization and structured synthesis (categories and common points) are preferred. 
    \item Examples: Gao et al. \cite{gao2024} reported an accuracy of XX\%, while van Gelderen and Tejedor-García \cite{vanGelderen2024} proposed a system that achieved an accuracy of YY\% and \cite{zhou2023comprehensive}...
    \item You might include subsections for specific subtopics, grouping similar papers results.
    \item You might reference tables or figures of the papers but do not include them in your manuscript.
\end{itemize}



\section{Discussion}
\begin{itemize}
    \item Discuss with your own words the implications of the results within the specific context of your topic (e.g., bias, explainability, application to real-world applications, etc.).
    \item Provide your personal insights and reflections on the results based on your short experience with speech technology and the course’s material (lectures, guest speakers and seminars/tutorials).
\end{itemize}




\section{Conclusion}

\begin{itemize}
    \item Summarize the key insights from your chosen topic.
    \item Highlight any gaps in current research and suggest avenues for future exploration.
    \item Personal thoughts about the course and the activities done in relation with the project are welcomed.
    \item No references are needed in this section.
\end{itemize}




% Cite in IEEE Transactions format all the papers/studies/datasets/foundation models referred in your manuscript.
% Include all details of each citation.
\bibliographystyle{IEEEtran}
\bibliography{mybib}

\end{document}